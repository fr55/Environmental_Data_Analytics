\documentclass[]{article}
\usepackage{lmodern}
\usepackage{amssymb,amsmath}
\usepackage{ifxetex,ifluatex}
\usepackage{fixltx2e} % provides \textsubscript
\ifnum 0\ifxetex 1\fi\ifluatex 1\fi=0 % if pdftex
  \usepackage[T1]{fontenc}
  \usepackage[utf8]{inputenc}
\else % if luatex or xelatex
  \ifxetex
    \usepackage{mathspec}
  \else
    \usepackage{fontspec}
  \fi
  \defaultfontfeatures{Ligatures=TeX,Scale=MatchLowercase}
\fi
% use upquote if available, for straight quotes in verbatim environments
\IfFileExists{upquote.sty}{\usepackage{upquote}}{}
% use microtype if available
\IfFileExists{microtype.sty}{%
\usepackage{microtype}
\UseMicrotypeSet[protrusion]{basicmath} % disable protrusion for tt fonts
}{}
\usepackage[margin=2.54cm]{geometry}
\usepackage{hyperref}
\hypersetup{unicode=true,
            pdftitle={4: Coding Basics},
            pdfauthor={Environmental Data Analytics \textbar{} Kateri Salk},
            pdfborder={0 0 0},
            breaklinks=true}
\urlstyle{same}  % don't use monospace font for urls
\usepackage{color}
\usepackage{fancyvrb}
\newcommand{\VerbBar}{|}
\newcommand{\VERB}{\Verb[commandchars=\\\{\}]}
\DefineVerbatimEnvironment{Highlighting}{Verbatim}{commandchars=\\\{\}}
% Add ',fontsize=\small' for more characters per line
\usepackage{framed}
\definecolor{shadecolor}{RGB}{248,248,248}
\newenvironment{Shaded}{\begin{snugshade}}{\end{snugshade}}
\newcommand{\KeywordTok}[1]{\textcolor[rgb]{0.13,0.29,0.53}{\textbf{#1}}}
\newcommand{\DataTypeTok}[1]{\textcolor[rgb]{0.13,0.29,0.53}{#1}}
\newcommand{\DecValTok}[1]{\textcolor[rgb]{0.00,0.00,0.81}{#1}}
\newcommand{\BaseNTok}[1]{\textcolor[rgb]{0.00,0.00,0.81}{#1}}
\newcommand{\FloatTok}[1]{\textcolor[rgb]{0.00,0.00,0.81}{#1}}
\newcommand{\ConstantTok}[1]{\textcolor[rgb]{0.00,0.00,0.00}{#1}}
\newcommand{\CharTok}[1]{\textcolor[rgb]{0.31,0.60,0.02}{#1}}
\newcommand{\SpecialCharTok}[1]{\textcolor[rgb]{0.00,0.00,0.00}{#1}}
\newcommand{\StringTok}[1]{\textcolor[rgb]{0.31,0.60,0.02}{#1}}
\newcommand{\VerbatimStringTok}[1]{\textcolor[rgb]{0.31,0.60,0.02}{#1}}
\newcommand{\SpecialStringTok}[1]{\textcolor[rgb]{0.31,0.60,0.02}{#1}}
\newcommand{\ImportTok}[1]{#1}
\newcommand{\CommentTok}[1]{\textcolor[rgb]{0.56,0.35,0.01}{\textit{#1}}}
\newcommand{\DocumentationTok}[1]{\textcolor[rgb]{0.56,0.35,0.01}{\textbf{\textit{#1}}}}
\newcommand{\AnnotationTok}[1]{\textcolor[rgb]{0.56,0.35,0.01}{\textbf{\textit{#1}}}}
\newcommand{\CommentVarTok}[1]{\textcolor[rgb]{0.56,0.35,0.01}{\textbf{\textit{#1}}}}
\newcommand{\OtherTok}[1]{\textcolor[rgb]{0.56,0.35,0.01}{#1}}
\newcommand{\FunctionTok}[1]{\textcolor[rgb]{0.00,0.00,0.00}{#1}}
\newcommand{\VariableTok}[1]{\textcolor[rgb]{0.00,0.00,0.00}{#1}}
\newcommand{\ControlFlowTok}[1]{\textcolor[rgb]{0.13,0.29,0.53}{\textbf{#1}}}
\newcommand{\OperatorTok}[1]{\textcolor[rgb]{0.81,0.36,0.00}{\textbf{#1}}}
\newcommand{\BuiltInTok}[1]{#1}
\newcommand{\ExtensionTok}[1]{#1}
\newcommand{\PreprocessorTok}[1]{\textcolor[rgb]{0.56,0.35,0.01}{\textit{#1}}}
\newcommand{\AttributeTok}[1]{\textcolor[rgb]{0.77,0.63,0.00}{#1}}
\newcommand{\RegionMarkerTok}[1]{#1}
\newcommand{\InformationTok}[1]{\textcolor[rgb]{0.56,0.35,0.01}{\textbf{\textit{#1}}}}
\newcommand{\WarningTok}[1]{\textcolor[rgb]{0.56,0.35,0.01}{\textbf{\textit{#1}}}}
\newcommand{\AlertTok}[1]{\textcolor[rgb]{0.94,0.16,0.16}{#1}}
\newcommand{\ErrorTok}[1]{\textcolor[rgb]{0.64,0.00,0.00}{\textbf{#1}}}
\newcommand{\NormalTok}[1]{#1}
\usepackage{graphicx,grffile}
\makeatletter
\def\maxwidth{\ifdim\Gin@nat@width>\linewidth\linewidth\else\Gin@nat@width\fi}
\def\maxheight{\ifdim\Gin@nat@height>\textheight\textheight\else\Gin@nat@height\fi}
\makeatother
% Scale images if necessary, so that they will not overflow the page
% margins by default, and it is still possible to overwrite the defaults
% using explicit options in \includegraphics[width, height, ...]{}
\setkeys{Gin}{width=\maxwidth,height=\maxheight,keepaspectratio}
\IfFileExists{parskip.sty}{%
\usepackage{parskip}
}{% else
\setlength{\parindent}{0pt}
\setlength{\parskip}{6pt plus 2pt minus 1pt}
}
\setlength{\emergencystretch}{3em}  % prevent overfull lines
\providecommand{\tightlist}{%
  \setlength{\itemsep}{0pt}\setlength{\parskip}{0pt}}
\setcounter{secnumdepth}{0}
% Redefines (sub)paragraphs to behave more like sections
\ifx\paragraph\undefined\else
\let\oldparagraph\paragraph
\renewcommand{\paragraph}[1]{\oldparagraph{#1}\mbox{}}
\fi
\ifx\subparagraph\undefined\else
\let\oldsubparagraph\subparagraph
\renewcommand{\subparagraph}[1]{\oldsubparagraph{#1}\mbox{}}
\fi

%%% Use protect on footnotes to avoid problems with footnotes in titles
\let\rmarkdownfootnote\footnote%
\def\footnote{\protect\rmarkdownfootnote}

%%% Change title format to be more compact
\usepackage{titling}

% Create subtitle command for use in maketitle
\newcommand{\subtitle}[1]{
  \posttitle{
    \begin{center}\large#1\end{center}
    }
}

\setlength{\droptitle}{-2em}

  \title{4: Coding Basics}
    \pretitle{\vspace{\droptitle}\centering\huge}
  \posttitle{\par}
    \author{Environmental Data Analytics \textbar{} Kateri Salk}
    \preauthor{\centering\large\emph}
  \postauthor{\par}
      \predate{\centering\large\emph}
  \postdate{\par}
    \date{Spring 2019}


\begin{document}
\maketitle

\subsection{LESSON OBJECTIVES}\label{lesson-objectives}

\begin{enumerate}
\def\labelenumi{\arabic{enumi}.}
\tightlist
\item
  Develop familiarity with the form and function of the RStudio
  interface.
\item
  Apply basic functionality of R
\item
  Evaluate how basic practies in R contribute to best management
  practices for data analysis
\end{enumerate}

\subsection{DATA TYPES IN R}\label{data-types-in-r}

R treats objects differently based on their characteristics. For more
information, please see:
\url{https://www.statmethods.net/input/datatypes.html}.

\begin{itemize}
\item
  \textbf{Vectors} 1 dimensional structure that contains elements of the
  same type.
\item
  \textbf{Matrices} 2 dimensional structure that contains elements of
  the same type.
\item
  \textbf{Arrays} Similar to matrices, but can have more than 2
  dimensions. We will not delve into arrays in depth.
\item
  \textbf{Lists} Ordered collection of elements that can have different
  modes.
\item
  \textbf{Data Frames} 2 dimensional structure that is more general than
  a matrix. Columns can have different modes (e.g., numeric and factor).
  When we import csv files into the R workspace, they will enter as data
  frames.
\end{itemize}

Define what each new piece of syntax does below (i.e., fill in blank
comments). Note that the R chunk has been divided into sections (\# at
beginning of line, ---- at end)

\begin{Shaded}
\begin{Highlighting}[]
\CommentTok{# Vectors ----}
\CommentTok{# the 4 - make a seccion}
\NormalTok{vector1 <-}\StringTok{ }\KeywordTok{c}\NormalTok{(}\DecValTok{1}\NormalTok{,}\DecValTok{2}\NormalTok{,}\FloatTok{5.3}\NormalTok{,}\DecValTok{6}\NormalTok{,}\OperatorTok{-}\DecValTok{2}\NormalTok{,}\DecValTok{4}\NormalTok{) }\CommentTok{# numeric vector}
\NormalTok{vector1}
\end{Highlighting}
\end{Shaded}

\begin{verbatim}
## [1]  1.0  2.0  5.3  6.0 -2.0  4.0
\end{verbatim}

\begin{Shaded}
\begin{Highlighting}[]
\NormalTok{vector2 <-}\StringTok{ }\KeywordTok{c}\NormalTok{(}\StringTok{"one"}\NormalTok{,}\StringTok{"two"}\NormalTok{,}\StringTok{"three"}\NormalTok{) }\CommentTok{# character vector}
\NormalTok{vector2}
\end{Highlighting}
\end{Shaded}

\begin{verbatim}
## [1] "one"   "two"   "three"
\end{verbatim}

\begin{Shaded}
\begin{Highlighting}[]
\NormalTok{vector3 <-}\StringTok{ }\KeywordTok{c}\NormalTok{(}\OtherTok{TRUE}\NormalTok{,}\OtherTok{TRUE}\NormalTok{,}\OtherTok{TRUE}\NormalTok{,}\OtherTok{FALSE}\NormalTok{,}\OtherTok{TRUE}\NormalTok{,}\OtherTok{FALSE}\NormalTok{) }\CommentTok{#logical vector}
\NormalTok{vector3}
\end{Highlighting}
\end{Shaded}

\begin{verbatim}
## [1]  TRUE  TRUE  TRUE FALSE  TRUE FALSE
\end{verbatim}

\begin{Shaded}
\begin{Highlighting}[]
\NormalTok{vector1[}\DecValTok{3}\NormalTok{] }\CommentTok{# it is getting the third element of vector 1. Called Matrix subsetting}
\end{Highlighting}
\end{Shaded}

\begin{verbatim}
## [1] 5.3
\end{verbatim}

\begin{Shaded}
\begin{Highlighting}[]
\CommentTok{# Matrices ----}
\NormalTok{matrix1 <-}\StringTok{ }\KeywordTok{matrix}\NormalTok{(}\DecValTok{1}\OperatorTok{:}\DecValTok{19}\NormalTok{, }\DataTypeTok{nrow =} \DecValTok{5}\NormalTok{, }\DataTypeTok{ncol =} \DecValTok{4}\NormalTok{) }\CommentTok{# numbers go vertically (default)}
\end{Highlighting}
\end{Shaded}

\begin{verbatim}
## Warning in matrix(1:19, nrow = 5, ncol = 4): la longitud de los datos [19]
## no es un submúltiplo o múltiplo del número de filas [5] en la matriz
\end{verbatim}

\begin{Shaded}
\begin{Highlighting}[]
\NormalTok{matrix1}
\end{Highlighting}
\end{Shaded}

\begin{verbatim}
##      [,1] [,2] [,3] [,4]
## [1,]    1    6   11   16
## [2,]    2    7   12   17
## [3,]    3    8   13   18
## [4,]    4    9   14   19
## [5,]    5   10   15    1
\end{verbatim}

\begin{Shaded}
\begin{Highlighting}[]
\NormalTok{matrix2 <-}\StringTok{ }\KeywordTok{matrix}\NormalTok{(}\DecValTok{1}\OperatorTok{:}\DecValTok{20}\NormalTok{, }\DataTypeTok{nrow =} \DecValTok{5}\NormalTok{, }\DataTypeTok{ncol =} \DecValTok{4}\NormalTok{, }\DataTypeTok{byrow =} \OtherTok{TRUE}\NormalTok{) }\CommentTok{# changes the deffault}
\NormalTok{matrix2}
\end{Highlighting}
\end{Shaded}

\begin{verbatim}
##      [,1] [,2] [,3] [,4]
## [1,]    1    2    3    4
## [2,]    5    6    7    8
## [3,]    9   10   11   12
## [4,]   13   14   15   16
## [5,]   17   18   19   20
\end{verbatim}

\begin{Shaded}
\begin{Highlighting}[]
\NormalTok{matrix3 <-}\StringTok{ }\KeywordTok{matrix}\NormalTok{(}\DecValTok{1}\OperatorTok{:}\DecValTok{20}\NormalTok{, }\DataTypeTok{nrow =} \DecValTok{5}\NormalTok{, }\DataTypeTok{ncol =} \DecValTok{4}\NormalTok{, }\DataTypeTok{byrow =} \OtherTok{TRUE}\NormalTok{, }\CommentTok{# return after comma continues the line}
                  \DataTypeTok{dimnames =} \KeywordTok{list}\NormalTok{(}\KeywordTok{c}\NormalTok{(}\StringTok{"uno"}\NormalTok{, }\StringTok{"dos"}\NormalTok{, }\StringTok{"tres"}\NormalTok{, }\StringTok{"cuatro"}\NormalTok{, }\StringTok{"cinco"}\NormalTok{), }
                                  \KeywordTok{c}\NormalTok{(}\StringTok{"un"}\NormalTok{, }\StringTok{"deux"}\NormalTok{, }\StringTok{"trois"}\NormalTok{, }\StringTok{"cat"}\NormalTok{))) }\CommentTok{#}
\NormalTok{matrix3}
\end{Highlighting}
\end{Shaded}

\begin{verbatim}
##        un deux trois cat
## uno     1    2     3   4
## dos     5    6     7   8
## tres    9   10    11  12
## cuatro 13   14    15  16
## cinco  17   18    19  20
\end{verbatim}

\begin{Shaded}
\begin{Highlighting}[]
\NormalTok{matrix1[}\DecValTok{4}\NormalTok{, ] }\CommentTok{# row 4 of matrix 1}
\end{Highlighting}
\end{Shaded}

\begin{verbatim}
## [1]  4  9 14 19
\end{verbatim}

\begin{Shaded}
\begin{Highlighting}[]
\NormalTok{matrix1[ , }\DecValTok{3}\NormalTok{] }\CommentTok{# }
\end{Highlighting}
\end{Shaded}

\begin{verbatim}
## [1] 11 12 13 14 15
\end{verbatim}

\begin{Shaded}
\begin{Highlighting}[]
\NormalTok{matrix1[}\KeywordTok{c}\NormalTok{(}\DecValTok{12}\NormalTok{, }\DecValTok{14}\NormalTok{)] }\CommentTok{#}
\end{Highlighting}
\end{Shaded}

\begin{verbatim}
## [1] 12 14
\end{verbatim}

\begin{Shaded}
\begin{Highlighting}[]
\NormalTok{matrix1[}\KeywordTok{c}\NormalTok{(}\DecValTok{12}\OperatorTok{:}\DecValTok{14}\NormalTok{)] }\CommentTok{#}
\end{Highlighting}
\end{Shaded}

\begin{verbatim}
## [1] 12 13 14
\end{verbatim}

\begin{Shaded}
\begin{Highlighting}[]
\NormalTok{matrix1[}\DecValTok{2}\OperatorTok{:}\DecValTok{4}\NormalTok{, }\DecValTok{1}\OperatorTok{:}\DecValTok{3}\NormalTok{] }\CommentTok{# matrix subset}
\end{Highlighting}
\end{Shaded}

\begin{verbatim}
##      [,1] [,2] [,3]
## [1,]    2    7   12
## [2,]    3    8   13
## [3,]    4    9   14
\end{verbatim}

\begin{Shaded}
\begin{Highlighting}[]
\NormalTok{cells <-}\StringTok{ }\KeywordTok{c}\NormalTok{(}\DecValTok{1}\NormalTok{, }\DecValTok{26}\NormalTok{, }\DecValTok{24}\NormalTok{, }\DecValTok{68}\NormalTok{)}
\NormalTok{rnames <-}\StringTok{ }\KeywordTok{c}\NormalTok{(}\StringTok{"R1"}\NormalTok{, }\StringTok{"R2"}\NormalTok{)}
\NormalTok{cnames <-}\StringTok{ }\KeywordTok{c}\NormalTok{(}\StringTok{"C1"}\NormalTok{, }\StringTok{"C2"}\NormalTok{) }
\NormalTok{matrix4 <-}\StringTok{ }\KeywordTok{matrix}\NormalTok{(cells, }\DataTypeTok{nrow =} \DecValTok{2}\NormalTok{, }\DataTypeTok{ncol =} \DecValTok{2}\NormalTok{, }\DataTypeTok{byrow =} \OtherTok{TRUE}\NormalTok{,}
  \DataTypeTok{dimnames =} \KeywordTok{list}\NormalTok{(rnames, cnames)) }\CommentTok{# }
\NormalTok{matrix4}
\end{Highlighting}
\end{Shaded}

\begin{verbatim}
##    C1 C2
## R1  1 26
## R2 24 68
\end{verbatim}

\begin{Shaded}
\begin{Highlighting}[]
\CommentTok{# Lists ---- }
\NormalTok{list1 <-}\StringTok{ }\KeywordTok{list}\NormalTok{(}\DataTypeTok{name =} \StringTok{"Maria"}\NormalTok{, }\DataTypeTok{mynumbers =}\NormalTok{ vector1, }\DataTypeTok{mymatrix =}\NormalTok{ matrix1, }\DataTypeTok{age =} \FloatTok{5.3}\NormalTok{); list1}
\end{Highlighting}
\end{Shaded}

\begin{verbatim}
## $name
## [1] "Maria"
## 
## $mynumbers
## [1]  1.0  2.0  5.3  6.0 -2.0  4.0
## 
## $mymatrix
##      [,1] [,2] [,3] [,4]
## [1,]    1    6   11   16
## [2,]    2    7   12   17
## [3,]    3    8   13   18
## [4,]    4    9   14   19
## [5,]    5   10   15    1
## 
## $age
## [1] 5.3
\end{verbatim}

\begin{Shaded}
\begin{Highlighting}[]
\NormalTok{list1[[}\DecValTok{2}\NormalTok{]]}
\end{Highlighting}
\end{Shaded}

\begin{verbatim}
## [1]  1.0  2.0  5.3  6.0 -2.0  4.0
\end{verbatim}

\begin{Shaded}
\begin{Highlighting}[]
\NormalTok{list1[[}\StringTok{"mynumbers"}\NormalTok{]] }\CommentTok{# the same as above}
\end{Highlighting}
\end{Shaded}

\begin{verbatim}
## [1]  1.0  2.0  5.3  6.0 -2.0  4.0
\end{verbatim}

\begin{Shaded}
\begin{Highlighting}[]
\CommentTok{# Data Frames ----}
\NormalTok{d <-}\StringTok{ }\KeywordTok{c}\NormalTok{(}\DecValTok{1}\NormalTok{, }\DecValTok{2}\NormalTok{, }\DecValTok{3}\NormalTok{, }\DecValTok{4}\NormalTok{) }\CommentTok{# What type of vector? Numeric}
\NormalTok{e <-}\StringTok{ }\KeywordTok{c}\NormalTok{(}\StringTok{"red"}\NormalTok{, }\StringTok{"white"}\NormalTok{, }\StringTok{"red"}\NormalTok{, }\OtherTok{NA}\NormalTok{) }\CommentTok{# What type of vector? Character}
\NormalTok{f <-}\StringTok{ }\KeywordTok{c}\NormalTok{(}\OtherTok{TRUE}\NormalTok{, }\OtherTok{TRUE}\NormalTok{, }\OtherTok{TRUE}\NormalTok{, }\OtherTok{FALSE}\NormalTok{) }\CommentTok{# What type of vector? Logical}
\NormalTok{dataframe1 <-}\StringTok{ }\KeywordTok{data.frame}\NormalTok{(d,e,f) }\CommentTok{#}
\KeywordTok{names}\NormalTok{(dataframe1) <-}\StringTok{ }\KeywordTok{c}\NormalTok{(}\StringTok{"ID"}\NormalTok{,}\StringTok{"Color"}\NormalTok{,}\StringTok{"Passed"}\NormalTok{); }\KeywordTok{View}\NormalTok{(dataframe1) }\CommentTok{# }

\NormalTok{dataframe1[}\DecValTok{1}\OperatorTok{:}\DecValTok{3}\NormalTok{] }\CommentTok{# you call columns 1 to 3. If you want rows put a coma afterwards}
\end{Highlighting}
\end{Shaded}

\begin{verbatim}
##   ID Color Passed
## 1  1   red   TRUE
## 2  2 white   TRUE
## 3  3   red   TRUE
## 4  4  <NA>  FALSE
\end{verbatim}

\begin{Shaded}
\begin{Highlighting}[]
\NormalTok{dataframe1[}\KeywordTok{c}\NormalTok{(}\StringTok{"ID"}\NormalTok{,}\StringTok{"Passed"}\NormalTok{)] }\CommentTok{# }
\end{Highlighting}
\end{Shaded}

\begin{verbatim}
##   ID Passed
## 1  1   TRUE
## 2  2   TRUE
## 3  3   TRUE
## 4  4  FALSE
\end{verbatim}

\begin{Shaded}
\begin{Highlighting}[]
\NormalTok{dataframe1}\OperatorTok{$}\NormalTok{Color }\CommentTok{# you get the column}
\end{Highlighting}
\end{Shaded}

\begin{verbatim}
## [1] red   white red   <NA> 
## Levels: red white
\end{verbatim}

QUESTION: How do the different types of data appear in the Environment
tab?

\begin{quote}
ANSWER: Data and values.
\end{quote}

QUESTION: In the R chunk below, write ``dataframe1\$''. Press
\texttt{tab} after you type the dollar sign. What happens?

\begin{quote}
ANSWER: you get the columns names in dataframe1
\end{quote}

QUESTION: What happens when a comment in R is followed by ``----''?

\begin{quote}
ANSWER: it generates a section that doesnt appear in your content list
\end{quote}

Advanced: Sequential section headers can be created by using at least
four -, =, and \# characters.

\subsection{PACKAGES}\label{packages}

The Packages tab in the notebook stores the packages that you have saved
in your system. A checkmark next to each package indicates whether the
package has been loaded into your current R session. Given that R is an
open source software, users can create packages that have specific
functionalities, with complicated code ``packaged'' into a simple
commands.

If you want to use a specific package that is not in your libaray
already, you need to install it. You can do this in two ways:

\begin{enumerate}
\def\labelenumi{\arabic{enumi}.}
\item
  Click the install button in the packages tab. Type the package name,
  which should autocomplete below (case matters). Make sure to check
  ``intall dependencies,'' which will also install packages that your
  new package uses.
\item
  Type \texttt{install.packages("packagename")} into your R chunk or
  console. It will then appear in your packages list. You only need to
  do this once.
\end{enumerate}

If a package is already installed, you will need to load it every
session. You can do this in two ways:

\begin{enumerate}
\def\labelenumi{\arabic{enumi}.}
\item
  Click the box next to the package name in the Packages tab.
\item
  Type \texttt{library(packagename)} into your R chunk or console.
\end{enumerate}

2a. The command \texttt{require(packagename)} will also load a package,
but it will not give any error or warning messages if there is an issue.

\textbf{Tips and troubleshooting}

\begin{itemize}
\item
  You may be asked to restart R when installing or updating packages.
  Feel free to say no, as this will obviously slow your progress.
  However, if the functionality of your new package isn't working
  properly, try restarting R as a first step.
\item
  If asked ``Do you want to install from sources the packages which
  needs compilation?'', type \texttt{yes} into the console.
\item
  You should only install packages once on your machine. If you store
  \texttt{install.packages} in your R chunks/scripts, comment these
  lines out, as below.
\item
  Update your packages regularly!
\end{itemize}

\begin{Shaded}
\begin{Highlighting}[]
\CommentTok{# We will use the packages dplyr and ggplot2 regularly. }
\CommentTok{#install.packages("dplyr") # comment out install commands, use only when needed and re-comment}
\CommentTok{#install.packages("ggplot2")}

\KeywordTok{library}\NormalTok{(dplyr)}
\end{Highlighting}
\end{Shaded}

\begin{verbatim}
## 
## Attaching package: 'dplyr'
\end{verbatim}

\begin{verbatim}
## The following objects are masked from 'package:stats':
## 
##     filter, lag
\end{verbatim}

\begin{verbatim}
## The following objects are masked from 'package:base':
## 
##     intersect, setdiff, setequal, union
\end{verbatim}

\begin{Shaded}
\begin{Highlighting}[]
\KeywordTok{library}\NormalTok{(ggplot2)}

\CommentTok{# Some packages are umbrellas under which other packages are loaded}
\CommentTok{#install.packages("tidyverse")}
\KeywordTok{library}\NormalTok{(tidyverse)}
\end{Highlighting}
\end{Shaded}

\begin{verbatim}
## -- Attaching packages ---------------------------------------------------- tidyverse 1.2.1 --
\end{verbatim}

\begin{verbatim}
## v tibble  1.4.2     v purrr   0.2.5
## v tidyr   0.8.1     v stringr 1.3.1
## v readr   1.1.1     v forcats 0.3.0
\end{verbatim}

\begin{verbatim}
## -- Conflicts ------------------------------------------------------- tidyverse_conflicts() --
## x dplyr::filter() masks stats::filter()
## x dplyr::lag()    masks stats::lag()
\end{verbatim}

What happens in the console when you load a package?

\begin{quote}
ANSWER: It says if you were succesful instaling the package and any
warnings.
\end{quote}

\subsection{FUNCTIONS}\label{functions}

You've had some practice with functions with the simple commands you've
entered in this lesson and the one previous. The basic form of a
function is \texttt{functionname()}, and the packages we will use in
this class will use these basic forms. However, there may be situations
when you will want to create your own function. Below is a description
of how to write functions through the metaphor of creating a recipe.
Credit for this goes to Isabella R. Ghement (@IsabellaGhement on
Twitter).

Writing a function is like writing a recipe. Your function will need a
recipe name (functionname). Your recipe ingredients will go inside the
parentheses. The recipe steps and end product go inside the curly
brackets.

\begin{Shaded}
\begin{Highlighting}[]
\NormalTok{functionname <-}\StringTok{ }\ControlFlowTok{function}\NormalTok{()\{}
  
\NormalTok{\}}
\CommentTok{#()ingredients \{\}recipe}
\end{Highlighting}
\end{Shaded}

A single ingredient recipe:

\begin{Shaded}
\begin{Highlighting}[]
\CommentTok{# Write the recipe}
\NormalTok{recipe1 <-}\StringTok{ }\ControlFlowTok{function}\NormalTok{(x)\{}
\NormalTok{  mix <-}\StringTok{ }\NormalTok{x}\OperatorTok{*}\DecValTok{2}
  \KeywordTok{return}\NormalTok{(mix)}
\NormalTok{\}}

\CommentTok{# Bake the recipe}
\NormalTok{simplemeal <-}\StringTok{ }\KeywordTok{recipe1}\NormalTok{(}\DecValTok{5}\NormalTok{)}

\CommentTok{# Serve the recipe}
\NormalTok{simplemeal}
\end{Highlighting}
\end{Shaded}

\begin{verbatim}
## [1] 10
\end{verbatim}

Two single ingredient recipes, baked at the same time:

\begin{Shaded}
\begin{Highlighting}[]
\NormalTok{recipe2 <-}\StringTok{ }\ControlFlowTok{function}\NormalTok{(x)\{}
\NormalTok{  mix1 <-}\StringTok{ }\NormalTok{x}\OperatorTok{*}\DecValTok{2}
\NormalTok{  mix2 <-}\StringTok{ }\NormalTok{x}\OperatorTok{/}\DecValTok{2}
  \KeywordTok{return}\NormalTok{(}\KeywordTok{list}\NormalTok{(}\DataTypeTok{mix1 =}\NormalTok{ mix1, }\CommentTok{#comma indicates we continue onto the next line}
              \DataTypeTok{mix2 =}\NormalTok{ mix2))}
\NormalTok{\}}

\NormalTok{doublesimplemeal <-}\StringTok{ }\KeywordTok{recipe2}\NormalTok{(}\DecValTok{6}\NormalTok{)}
\NormalTok{doublesimplemeal}
\end{Highlighting}
\end{Shaded}

\begin{verbatim}
## $mix1
## [1] 12
## 
## $mix2
## [1] 3
\end{verbatim}

Two double ingredient recipes, baked at the same time:

\begin{Shaded}
\begin{Highlighting}[]
\NormalTok{recipe3 <-}\StringTok{ }\ControlFlowTok{function}\NormalTok{(x, f)\{}
\NormalTok{  mix1 <-}\StringTok{ }\NormalTok{x}\OperatorTok{*}\NormalTok{f}
\NormalTok{  mix2 <-}\StringTok{ }\NormalTok{x}\OperatorTok{/}\NormalTok{f}
  \KeywordTok{return}\NormalTok{(}\KeywordTok{list}\NormalTok{(}\DataTypeTok{mix1 =}\NormalTok{ mix1, }\CommentTok{#comma indicates we continue onto the next line}
              \DataTypeTok{mix2 =}\NormalTok{ mix2))}
\NormalTok{\}}

\NormalTok{doublecomplexmeal <-}\StringTok{ }\KeywordTok{recipe3}\NormalTok{(}\DataTypeTok{x =} \DecValTok{5}\NormalTok{, }\DataTypeTok{f =} \DecValTok{2}\NormalTok{)}
\NormalTok{doublecomplexmeal}
\end{Highlighting}
\end{Shaded}

\begin{verbatim}
## $mix1
## [1] 10
## 
## $mix2
## [1] 2.5
\end{verbatim}

\begin{Shaded}
\begin{Highlighting}[]
\NormalTok{doublecomplexmeal}\OperatorTok{$}\NormalTok{mix1}
\end{Highlighting}
\end{Shaded}

\begin{verbatim}
## [1] 10
\end{verbatim}

Make a recipe based on the ingredients you have

\begin{Shaded}
\begin{Highlighting}[]
\NormalTok{recipe4 <-}\StringTok{ }\ControlFlowTok{function}\NormalTok{(x) \{}
  \ControlFlowTok{if}\NormalTok{(x }\OperatorTok{<}\StringTok{ }\DecValTok{3}\NormalTok{) \{}
\NormalTok{    x}\OperatorTok{*}\DecValTok{2}
\NormalTok{  \} }
  \ControlFlowTok{else}\NormalTok{ \{}
\NormalTok{    x}\OperatorTok{/}\DecValTok{2}
\NormalTok{  \}}
\NormalTok{\}}
\CommentTok{# if else can be used with vectors}
\NormalTok{recipe5 <-}\StringTok{ }\ControlFlowTok{function}\NormalTok{(x) \{}
  \ControlFlowTok{if}\NormalTok{(x }\OperatorTok{<}\StringTok{ }\DecValTok{3}\NormalTok{) \{}
\NormalTok{    x}\OperatorTok{*}\DecValTok{2}
\NormalTok{  \} }
  \ControlFlowTok{else} \ControlFlowTok{if}\NormalTok{ (x }\OperatorTok{>}\StringTok{ }\DecValTok{3}\NormalTok{) \{}
\NormalTok{    x}\OperatorTok{/}\DecValTok{2}
\NormalTok{  \}}
  \ControlFlowTok{else}\NormalTok{ \{}
\NormalTok{    x}
\NormalTok{  \}}
\NormalTok{\}}
\NormalTok{meal <-}\StringTok{ }\KeywordTok{recipe4}\NormalTok{(}\DecValTok{4}\NormalTok{); meal}
\end{Highlighting}
\end{Shaded}

\begin{verbatim}
## [1] 2
\end{verbatim}

\begin{Shaded}
\begin{Highlighting}[]
\NormalTok{meal2 <-}\StringTok{ }\KeywordTok{recipe4}\NormalTok{(}\DecValTok{2}\NormalTok{); meal2}
\end{Highlighting}
\end{Shaded}

\begin{verbatim}
## [1] 4
\end{verbatim}

\begin{Shaded}
\begin{Highlighting}[]
\NormalTok{meal3 <-}\StringTok{ }\KeywordTok{recipe5}\NormalTok{(}\DecValTok{3}\NormalTok{); meal3}
\end{Highlighting}
\end{Shaded}

\begin{verbatim}
## [1] 3
\end{verbatim}

\begin{Shaded}
\begin{Highlighting}[]
\NormalTok{recipe6 <-}\StringTok{ }\ControlFlowTok{function}\NormalTok{(x)\{}
  \KeywordTok{ifelse}\NormalTok{(x}\OperatorTok{<}\DecValTok{3}\NormalTok{, x}\OperatorTok{*}\DecValTok{2}\NormalTok{, x}\OperatorTok{/}\DecValTok{2}\NormalTok{)}
  
\NormalTok{\}}

\NormalTok{meal4 <-}\StringTok{ }\KeywordTok{recipe6}\NormalTok{(}\DecValTok{4}\NormalTok{); meal4}
\end{Highlighting}
\end{Shaded}

\begin{verbatim}
## [1] 2
\end{verbatim}

\begin{Shaded}
\begin{Highlighting}[]
\NormalTok{meal5 <-}\StringTok{ }\KeywordTok{recipe6}\NormalTok{(}\DecValTok{2}\NormalTok{); meal5}
\end{Highlighting}
\end{Shaded}

\begin{verbatim}
## [1] 4
\end{verbatim}


\end{document}
