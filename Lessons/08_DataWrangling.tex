\documentclass[]{article}
\usepackage{lmodern}
\usepackage{amssymb,amsmath}
\usepackage{ifxetex,ifluatex}
\usepackage{fixltx2e} % provides \textsubscript
\ifnum 0\ifxetex 1\fi\ifluatex 1\fi=0 % if pdftex
  \usepackage[T1]{fontenc}
  \usepackage[utf8]{inputenc}
\else % if luatex or xelatex
  \ifxetex
    \usepackage{mathspec}
  \else
    \usepackage{fontspec}
  \fi
  \defaultfontfeatures{Ligatures=TeX,Scale=MatchLowercase}
\fi
% use upquote if available, for straight quotes in verbatim environments
\IfFileExists{upquote.sty}{\usepackage{upquote}}{}
% use microtype if available
\IfFileExists{microtype.sty}{%
\usepackage{microtype}
\UseMicrotypeSet[protrusion]{basicmath} % disable protrusion for tt fonts
}{}
\usepackage[margin=2.54cm]{geometry}
\usepackage{hyperref}
\hypersetup{unicode=true,
            pdftitle={8: Data Wrangling},
            pdfauthor={Environmental Data Analytics \textbar{} Kateri Salk},
            pdfborder={0 0 0},
            breaklinks=true}
\urlstyle{same}  % don't use monospace font for urls
\usepackage{color}
\usepackage{fancyvrb}
\newcommand{\VerbBar}{|}
\newcommand{\VERB}{\Verb[commandchars=\\\{\}]}
\DefineVerbatimEnvironment{Highlighting}{Verbatim}{commandchars=\\\{\}}
% Add ',fontsize=\small' for more characters per line
\usepackage{framed}
\definecolor{shadecolor}{RGB}{248,248,248}
\newenvironment{Shaded}{\begin{snugshade}}{\end{snugshade}}
\newcommand{\KeywordTok}[1]{\textcolor[rgb]{0.13,0.29,0.53}{\textbf{#1}}}
\newcommand{\DataTypeTok}[1]{\textcolor[rgb]{0.13,0.29,0.53}{#1}}
\newcommand{\DecValTok}[1]{\textcolor[rgb]{0.00,0.00,0.81}{#1}}
\newcommand{\BaseNTok}[1]{\textcolor[rgb]{0.00,0.00,0.81}{#1}}
\newcommand{\FloatTok}[1]{\textcolor[rgb]{0.00,0.00,0.81}{#1}}
\newcommand{\ConstantTok}[1]{\textcolor[rgb]{0.00,0.00,0.00}{#1}}
\newcommand{\CharTok}[1]{\textcolor[rgb]{0.31,0.60,0.02}{#1}}
\newcommand{\SpecialCharTok}[1]{\textcolor[rgb]{0.00,0.00,0.00}{#1}}
\newcommand{\StringTok}[1]{\textcolor[rgb]{0.31,0.60,0.02}{#1}}
\newcommand{\VerbatimStringTok}[1]{\textcolor[rgb]{0.31,0.60,0.02}{#1}}
\newcommand{\SpecialStringTok}[1]{\textcolor[rgb]{0.31,0.60,0.02}{#1}}
\newcommand{\ImportTok}[1]{#1}
\newcommand{\CommentTok}[1]{\textcolor[rgb]{0.56,0.35,0.01}{\textit{#1}}}
\newcommand{\DocumentationTok}[1]{\textcolor[rgb]{0.56,0.35,0.01}{\textbf{\textit{#1}}}}
\newcommand{\AnnotationTok}[1]{\textcolor[rgb]{0.56,0.35,0.01}{\textbf{\textit{#1}}}}
\newcommand{\CommentVarTok}[1]{\textcolor[rgb]{0.56,0.35,0.01}{\textbf{\textit{#1}}}}
\newcommand{\OtherTok}[1]{\textcolor[rgb]{0.56,0.35,0.01}{#1}}
\newcommand{\FunctionTok}[1]{\textcolor[rgb]{0.00,0.00,0.00}{#1}}
\newcommand{\VariableTok}[1]{\textcolor[rgb]{0.00,0.00,0.00}{#1}}
\newcommand{\ControlFlowTok}[1]{\textcolor[rgb]{0.13,0.29,0.53}{\textbf{#1}}}
\newcommand{\OperatorTok}[1]{\textcolor[rgb]{0.81,0.36,0.00}{\textbf{#1}}}
\newcommand{\BuiltInTok}[1]{#1}
\newcommand{\ExtensionTok}[1]{#1}
\newcommand{\PreprocessorTok}[1]{\textcolor[rgb]{0.56,0.35,0.01}{\textit{#1}}}
\newcommand{\AttributeTok}[1]{\textcolor[rgb]{0.77,0.63,0.00}{#1}}
\newcommand{\RegionMarkerTok}[1]{#1}
\newcommand{\InformationTok}[1]{\textcolor[rgb]{0.56,0.35,0.01}{\textbf{\textit{#1}}}}
\newcommand{\WarningTok}[1]{\textcolor[rgb]{0.56,0.35,0.01}{\textbf{\textit{#1}}}}
\newcommand{\AlertTok}[1]{\textcolor[rgb]{0.94,0.16,0.16}{#1}}
\newcommand{\ErrorTok}[1]{\textcolor[rgb]{0.64,0.00,0.00}{\textbf{#1}}}
\newcommand{\NormalTok}[1]{#1}
\usepackage{graphicx,grffile}
\makeatletter
\def\maxwidth{\ifdim\Gin@nat@width>\linewidth\linewidth\else\Gin@nat@width\fi}
\def\maxheight{\ifdim\Gin@nat@height>\textheight\textheight\else\Gin@nat@height\fi}
\makeatother
% Scale images if necessary, so that they will not overflow the page
% margins by default, and it is still possible to overwrite the defaults
% using explicit options in \includegraphics[width, height, ...]{}
\setkeys{Gin}{width=\maxwidth,height=\maxheight,keepaspectratio}
\IfFileExists{parskip.sty}{%
\usepackage{parskip}
}{% else
\setlength{\parindent}{0pt}
\setlength{\parskip}{6pt plus 2pt minus 1pt}
}
\setlength{\emergencystretch}{3em}  % prevent overfull lines
\providecommand{\tightlist}{%
  \setlength{\itemsep}{0pt}\setlength{\parskip}{0pt}}
\setcounter{secnumdepth}{0}
% Redefines (sub)paragraphs to behave more like sections
\ifx\paragraph\undefined\else
\let\oldparagraph\paragraph
\renewcommand{\paragraph}[1]{\oldparagraph{#1}\mbox{}}
\fi
\ifx\subparagraph\undefined\else
\let\oldsubparagraph\subparagraph
\renewcommand{\subparagraph}[1]{\oldsubparagraph{#1}\mbox{}}
\fi

%%% Use protect on footnotes to avoid problems with footnotes in titles
\let\rmarkdownfootnote\footnote%
\def\footnote{\protect\rmarkdownfootnote}

%%% Change title format to be more compact
\usepackage{titling}

% Create subtitle command for use in maketitle
\newcommand{\subtitle}[1]{
  \posttitle{
    \begin{center}\large#1\end{center}
    }
}

\setlength{\droptitle}{-2em}

  \title{8: Data Wrangling}
    \pretitle{\vspace{\droptitle}\centering\huge}
  \posttitle{\par}
    \author{Environmental Data Analytics \textbar{} Kateri Salk}
    \preauthor{\centering\large\emph}
  \postauthor{\par}
      \predate{\centering\large\emph}
  \postdate{\par}
    \date{Spring 2019}


\begin{document}
\maketitle

\subsection{LESSON OBJECTIVES}\label{lesson-objectives}

\begin{enumerate}
\def\labelenumi{\arabic{enumi}.}
\tightlist
\item
  Describe the usefulness of data wrangling and its place in the data
  pipeline
\item
  Wrangle datasets with dplyr functions
\item
  Apply data wrangling skills to a real-world example dataset
\end{enumerate}

\subsection{SET UP YOUR DATA ANALYSIS
SESSION}\label{set-up-your-data-analysis-session}

\begin{Shaded}
\begin{Highlighting}[]
\KeywordTok{getwd}\NormalTok{()}
\end{Highlighting}
\end{Shaded}

\begin{verbatim}
## [1] "C:/Users/Felipe/OneDrive - Duke University/1. DUKE/1. Ramos 2 Semestre/EOS-872 Env. Data Analytics/Environmental_Data_Analytics"
\end{verbatim}

\begin{Shaded}
\begin{Highlighting}[]
\KeywordTok{library}\NormalTok{(tidyverse)}
\end{Highlighting}
\end{Shaded}

\begin{verbatim}
## -- Attaching packages ---------------------------------------------------- tidyverse 1.2.1 --
\end{verbatim}

\begin{verbatim}
## v ggplot2 3.0.0     v purrr   0.2.5
## v tibble  1.4.2     v dplyr   0.7.6
## v tidyr   0.8.1     v stringr 1.3.1
## v readr   1.1.1     v forcats 0.3.0
\end{verbatim}

\begin{verbatim}
## -- Conflicts ------------------------------------------------------- tidyverse_conflicts() --
## x dplyr::filter() masks stats::filter()
## x dplyr::lag()    masks stats::lag()
\end{verbatim}

\begin{Shaded}
\begin{Highlighting}[]
\NormalTok{NTL.phys.data.PeterPaul <-}\StringTok{ }\KeywordTok{read.csv}\NormalTok{(}\StringTok{"./Data/Processed/NTL-LTER_Lake_ChemistryPhysics_PeterPaul_Processed.csv"}\NormalTok{)}
\NormalTok{NTL.nutrient.data <-}\StringTok{ }\KeywordTok{read.csv}\NormalTok{(}\StringTok{"./Data/Raw/NTL-LTER_Lake_Nutrients_Raw.csv"}\NormalTok{)}
\end{Highlighting}
\end{Shaded}

\subsection{REVIEW OF BASIC DATA EXPLORATION AND
WRANGLING}\label{review-of-basic-data-exploration-and-wrangling}

\begin{Shaded}
\begin{Highlighting}[]
\CommentTok{# Data summaries for physical data}
\KeywordTok{head}\NormalTok{(NTL.phys.data.PeterPaul)}
\end{Highlighting}
\end{Shaded}

\begin{verbatim}
##   lakeid  lakename year4 daynum sampledate depth temperature_C
## 1      L Paul Lake  1984    148    5/27/84  0.00          14.5
## 2      L Paul Lake  1984    148    5/27/84  0.25            NA
## 3      L Paul Lake  1984    148    5/27/84  0.50            NA
## 4      L Paul Lake  1984    148    5/27/84  0.75            NA
## 5      L Paul Lake  1984    148    5/27/84  1.00          14.5
## 6      L Paul Lake  1984    148    5/27/84  1.50            NA
##   dissolvedOxygen irradianceWater irradianceDeck comments
## 1             9.5            1750           1620     <NA>
## 2              NA            1550           1620     <NA>
## 3              NA            1150           1620     <NA>
## 4              NA             975           1620     <NA>
## 5             8.8             870           1620     <NA>
## 6              NA             610           1620     <NA>
\end{verbatim}

\begin{Shaded}
\begin{Highlighting}[]
\KeywordTok{colnames}\NormalTok{(NTL.phys.data.PeterPaul)}
\end{Highlighting}
\end{Shaded}

\begin{verbatim}
##  [1] "lakeid"          "lakename"        "year4"          
##  [4] "daynum"          "sampledate"      "depth"          
##  [7] "temperature_C"   "dissolvedOxygen" "irradianceWater"
## [10] "irradianceDeck"  "comments"
\end{verbatim}

\begin{Shaded}
\begin{Highlighting}[]
\KeywordTok{dim}\NormalTok{(NTL.phys.data.PeterPaul)}
\end{Highlighting}
\end{Shaded}

\begin{verbatim}
## [1] 21613    11
\end{verbatim}

\begin{Shaded}
\begin{Highlighting}[]
\KeywordTok{summary}\NormalTok{(NTL.phys.data.PeterPaul}\OperatorTok{$}\NormalTok{comments)}
\end{Highlighting}
\end{Shaded}

\begin{verbatim}
## DO Probe bad - Doesn't go to zero     DO taken with Jones Lab Meter 
##                               132                               112 
##                              NA's 
##                             21369
\end{verbatim}

\begin{Shaded}
\begin{Highlighting}[]
\KeywordTok{class}\NormalTok{(NTL.phys.data.PeterPaul}\OperatorTok{$}\NormalTok{sampledate) }\CommentTok{# CSV doesnt keep format info}
\end{Highlighting}
\end{Shaded}

\begin{verbatim}
## [1] "factor"
\end{verbatim}

\begin{Shaded}
\begin{Highlighting}[]
\CommentTok{# Format sampledate as date}
\NormalTok{NTL.phys.data.PeterPaul}\OperatorTok{$}\NormalTok{sampledate <-}\StringTok{ }\KeywordTok{as.Date}\NormalTok{(NTL.phys.data.PeterPaul}\OperatorTok{$}\NormalTok{sampledate, }\DataTypeTok{format =} \StringTok{"%m/%d/%y"}\NormalTok{)}

\CommentTok{# Select Peter and Paul Lakes from the nutrient dataset}
\NormalTok{NTL.nutrient.data.PeterPaul <-}\StringTok{ }\KeywordTok{filter}\NormalTok{(NTL.nutrient.data, lakename }\OperatorTok{==}\StringTok{ "Paul Lake"} \OperatorTok{|}\StringTok{ }\NormalTok{lakename }\OperatorTok{==}\StringTok{ "Peter Lake"}\NormalTok{) }\CommentTok{#filter rows, select colums. | = to or.}

\CommentTok{# Data summaries for nutrient data}
\KeywordTok{head}\NormalTok{(NTL.nutrient.data.PeterPaul)}
\end{Highlighting}
\end{Shaded}

\begin{verbatim}
##   lakeid  lakename year4 daynum sampledate depth_id depth tn_ug tp_ug nh34
## 1      L Paul Lake  1991    140    5/20/91        1  0.00   538    25   NA
## 2      L Paul Lake  1991    140    5/20/91        2  0.85   285    14   NA
## 3      L Paul Lake  1991    140    5/20/91        3  1.75   399    14   NA
## 4      L Paul Lake  1991    140    5/20/91        4  3.00   453    14   NA
## 5      L Paul Lake  1991    140    5/20/91        5  4.00   363    13   NA
## 6      L Paul Lake  1991    140    5/20/91        6  6.00   583    37   NA
##   no23 po4 comments
## 1   NA  NA         
## 2   NA  NA         
## 3   NA  NA         
## 4   NA  NA         
## 5   NA  NA         
## 6   NA  NA
\end{verbatim}

\begin{Shaded}
\begin{Highlighting}[]
\KeywordTok{colnames}\NormalTok{(NTL.nutrient.data.PeterPaul)}
\end{Highlighting}
\end{Shaded}

\begin{verbatim}
##  [1] "lakeid"     "lakename"   "year4"      "daynum"     "sampledate"
##  [6] "depth_id"   "depth"      "tn_ug"      "tp_ug"      "nh34"      
## [11] "no23"       "po4"        "comments"
\end{verbatim}

\begin{Shaded}
\begin{Highlighting}[]
\KeywordTok{dim}\NormalTok{(NTL.nutrient.data.PeterPaul)}
\end{Highlighting}
\end{Shaded}

\begin{verbatim}
## [1] 2770   13
\end{verbatim}

\begin{Shaded}
\begin{Highlighting}[]
\KeywordTok{summary}\NormalTok{(NTL.nutrient.data.PeterPaul}\OperatorTok{$}\NormalTok{comments) }\CommentTok{# the 0 comments are comments that exist for other lakes.}
\end{Highlighting}
\end{Shaded}

\begin{verbatim}
##                                                sample missing 
##                           2770                              0 
## TP value suspect, far too high 
##                              0
\end{verbatim}

\begin{Shaded}
\begin{Highlighting}[]
\KeywordTok{class}\NormalTok{(NTL.nutrient.data.PeterPaul}\OperatorTok{$}\NormalTok{comments) }\CommentTok{# is a factor. doesn not fill it with NAs.}
\end{Highlighting}
\end{Shaded}

\begin{verbatim}
## [1] "factor"
\end{verbatim}

\begin{Shaded}
\begin{Highlighting}[]
\KeywordTok{class}\NormalTok{(NTL.nutrient.data.PeterPaul}\OperatorTok{$}\NormalTok{sampledate)}
\end{Highlighting}
\end{Shaded}

\begin{verbatim}
## [1] "factor"
\end{verbatim}

\begin{Shaded}
\begin{Highlighting}[]
\NormalTok{NTL.nutrient.data.PeterPaul}\OperatorTok{$}\NormalTok{sampledate <-}\StringTok{ }\KeywordTok{as.Date}\NormalTok{(NTL.nutrient.data.PeterPaul}\OperatorTok{$}\NormalTok{sampledate, }\DataTypeTok{format =} \StringTok{"%m/%d/%y"}\NormalTok{)}

\KeywordTok{summary}\NormalTok{(NTL.nutrient.data.PeterPaul}\OperatorTok{$}\NormalTok{lakename) }\CommentTok{# shows all lakes. there is a droplevel that shows only the lakes in the data set.}
\end{Highlighting}
\end{Shaded}

\begin{verbatim}
##       Bergner Lake            Bog Pot         Bolger Bog 
##                  0                  0                  0 
##         Brown Lake  Central Long Lake      Crampton Lake 
##                  0                  0                  0 
##      Cranberry Bog     East Long Lake            Eds Bog 
##                  0                  0                  0 
## Forest Service Bog   Hummingbird Lake        Inkpot Lake 
##                  0                  0                  0 
##      Kickapoo Lake        Morris Lake     North Gate Bog 
##                  0                  0                  0 
##          Paul Lake         Peter Lake          Plum Lake 
##               1383               1387                  0 
##     Raspberry Lake    Reddington Lake         Roach Lake 
##                  0                  0                  0 
##         Tender Bog    Tenderfoot Lake       Tuesday Lake 
##                  0                  0                  0 
##          Ward Lake     West Long Lake 
##                  0                  0
\end{verbatim}

\begin{Shaded}
\begin{Highlighting}[]
\CommentTok{# Save processed nutrient file}
\KeywordTok{write.csv}\NormalTok{(NTL.nutrient.data.PeterPaul, }\DataTypeTok{row.names =} \OtherTok{FALSE}\NormalTok{, }\DataTypeTok{file =} \StringTok{"./Data/Processed/NTL-LTER_Lake_Nutrients_PeterPaul_Processed.csv"}\NormalTok{)}

\CommentTok{# Remove columns that are not of interest for analysis}
\NormalTok{NTL.phys.data.PeterPaul.skinny <-}\StringTok{ }\KeywordTok{select}\NormalTok{(NTL.phys.data.PeterPaul, }
\NormalTok{                                         lakename, daynum, year4, sampledate}\OperatorTok{:}\NormalTok{irradianceDeck)}
  
\NormalTok{NTL.nutrient.data.PeterPaul.skinny <-}\StringTok{ }\KeywordTok{select}\NormalTok{(NTL.nutrient.data.PeterPaul, }
\NormalTok{                                             lakename, daynum, year4, sampledate, depth}\OperatorTok{:}\NormalTok{po4)   }\CommentTok{# to remove we put a - sign (only for select). filter is different}
\end{Highlighting}
\end{Shaded}

\section{up to here is data
wrangling}\label{up-to-here-is-data-wrangling}

\subsection{TIDY DATASETS}\label{tidy-datasets}

For most situations, data analysis works best when you have organized
your data into a tidy dataset. A tidy dataset is defined as:

\begin{itemize}
\tightlist
\item
  Each variable is a column
\item
  Each row is an observation (e.g., sampling event from a specific date
  and/or location)
\item
  Each value is in its own cell
\end{itemize}

However, there may be situations where we want to reshape our dataset,
for example if we want to facet numerical data points by measurement
type (more on this in the data visualization unit). We can program this
reshaping in a few short lines of code using the package \texttt{tidyr},
which is conveniently included in the \texttt{tidyverse} package.

\begin{Shaded}
\begin{Highlighting}[]
\CommentTok{# Gather nutrient data into one column}
\NormalTok{NTL.nutrient.data.PeterPaul.gathered <-}\StringTok{ }\KeywordTok{gather}\NormalTok{(NTL.nutrient.data.PeterPaul.skinny, }\StringTok{"nutrient"}\NormalTok{, }\StringTok{"concentration"}\NormalTok{, tn_ug}\OperatorTok{:}\NormalTok{po4) }\CommentTok{# function gather and spread. we define new #columns and gather existing ones}
\KeywordTok{count}\NormalTok{(NTL.nutrient.data.PeterPaul.gathered, nutrient) }\CommentTok{# gets the bigger number and fills with NAs}
\end{Highlighting}
\end{Shaded}

\begin{verbatim}
## # A tibble: 5 x 2
##   nutrient     n
##   <chr>    <int>
## 1 nh34      2770
## 2 no23      2770
## 3 po4       2770
## 4 tn_ug     2770
## 5 tp_ug     2770
\end{verbatim}

\begin{Shaded}
\begin{Highlighting}[]
\NormalTok{NTL.nutrient.data.PeterPaul.gathered <-}\StringTok{ }\KeywordTok{subset}\NormalTok{(NTL.nutrient.data.PeterPaul.gathered, }\OperatorTok{!}\KeywordTok{is.na}\NormalTok{(concentration))}
\KeywordTok{count}\NormalTok{(NTL.nutrient.data.PeterPaul.gathered, nutrient) }\CommentTok{# eliminates the NAs}
\end{Highlighting}
\end{Shaded}

\begin{verbatim}
## # A tibble: 5 x 2
##   nutrient     n
##   <chr>    <int>
## 1 nh34      1204
## 2 no23      1235
## 3 po4       1246
## 4 tn_ug     1729
## 5 tp_ug     2583
\end{verbatim}

\begin{Shaded}
\begin{Highlighting}[]
\KeywordTok{write.csv}\NormalTok{(NTL.nutrient.data.PeterPaul.gathered, }\DataTypeTok{row.names =} \OtherTok{FALSE}\NormalTok{, }
          \DataTypeTok{file =}\StringTok{"./Data/Processed/NTL-LTER_Lake_Nutrients_PeterPaulGathered_Processed.csv"}\NormalTok{)}

\CommentTok{# Spread nutrient data into separate columns   # opposite of gather}
\NormalTok{NTL.nutrient.data.PeterPaul.spread <-}\StringTok{ }\KeywordTok{spread}\NormalTok{(NTL.nutrient.data.PeterPaul.gathered, nutrient, concentration) }\CommentTok{# two values missing. must be 2 rows with only NAs.}

\CommentTok{# Split components of cells into multiple columns}
\CommentTok{# Opposite of 'separate' is 'unite'}
\NormalTok{NTL.nutrient.data.PeterPaul.dates <-}\StringTok{ }\KeywordTok{separate}\NormalTok{(NTL.nutrient.data.PeterPaul.skinny, sampledate, }\KeywordTok{c}\NormalTok{(}\StringTok{"Y"}\NormalTok{, }\StringTok{"m"}\NormalTok{, }\StringTok{"d"}\NormalTok{))}
\end{Highlighting}
\end{Shaded}

\subsection{JOINING MULTIPLE DATASETS}\label{joining-multiple-datasets}

In many cases, we will want to combine datasets into one dataset. If all
column names match, the data frames can be combined with the
\texttt{rbind} function. If some column names match and some column
names don't match, we can combine the data frames using a ``join''
function according to common conditions that exist in the matching
columns. We will demonstrate this with the NTL-LTER physical and
nutrient datasets, where we have specific instances when physical and
nutrient data were collected on the same date, at the same lake, and at
the same depth.

In dplyr, there are several types of join functions:

\begin{itemize}
\tightlist
\item
  \texttt{inner\_join}: return rows in x where there are matching values
  in y, and all columns in x and y (mutating join).
\item
  \texttt{semi\_join}: return all rows from x where there are matching
  values in y, keeping just columns from x (filtering join).
\item
  \texttt{left\_join}: return all rows from x, and all columns from x
  and y (mutating join).
\item
  \texttt{anti\_join}: return all rows from x where there are \emph{not}
  matching values in y, keeping just columns from x (filtering join).
\item
  \texttt{full\_join}: return all rows and all columns from x and y.
  Returns NA for missing values (mutating join).
\end{itemize}

Let's say we want to generate a new dataset that contains all possible
physical and chemical data for Peter and Paul Lakes. In this case, we
want to do a full join.

\begin{Shaded}
\begin{Highlighting}[]
\NormalTok{NTL.phys.nutrient.data.PeterPaul <-}\StringTok{ }\KeywordTok{full_join}\NormalTok{(NTL.phys.data.PeterPaul.skinny, NTL.nutrient.data.PeterPaul.skinny)}
\end{Highlighting}
\end{Shaded}

\begin{verbatim}
## Joining, by = c("lakename", "daynum", "year4", "sampledate", "depth")
\end{verbatim}

\begin{verbatim}
## Warning: Column `lakename` joining factors with different levels, coercing
## to character vector
\end{verbatim}

\begin{Shaded}
\begin{Highlighting}[]
\KeywordTok{write.csv}\NormalTok{(NTL.phys.nutrient.data.PeterPaul, }\DataTypeTok{row.names =} \OtherTok{FALSE}\NormalTok{, }
          \DataTypeTok{file =}\StringTok{"./Data/Processed/NTL-LTER_Lake_Chemistry_Nutrients_PeterPaul_Processed.csv"}\NormalTok{)}
\CommentTok{# warning: switching to factor}
\end{Highlighting}
\end{Shaded}

\subsection{LUBRIDATE}\label{lubridate}

A package that makes coercing date much easier is \texttt{lubridate}. A
guide to the package can be found at
\url{https://lubridate.tidyverse.org/}. The cheat sheet within that web
page is excellent too. This package can do many things (hint: look into
this package if you are having unique date-type issues), but today we
will be using two of its functions for our NTL dataset.

\begin{Shaded}
\begin{Highlighting}[]
\CommentTok{#install.packages("lubridate")}
\KeywordTok{library}\NormalTok{(lubridate)}
\end{Highlighting}
\end{Shaded}

\begin{verbatim}
## 
## Attaching package: 'lubridate'
\end{verbatim}

\begin{verbatim}
## The following object is masked from 'package:base':
## 
##     date
\end{verbatim}

\begin{Shaded}
\begin{Highlighting}[]
\CommentTok{# add a month column to the dataset}
\NormalTok{NTL.phys.nutrient.data.PeterPaul <-}\StringTok{ }\KeywordTok{mutate}\NormalTok{(NTL.phys.nutrient.data.PeterPaul, }\DataTypeTok{month =} \KeywordTok{month}\NormalTok{(sampledate)) }

\CommentTok{# reorder columns to put month with the rest of the date variables}
\NormalTok{NTL.phys.nutrient.data.PeterPaul <-}\StringTok{ }\KeywordTok{select}\NormalTok{(NTL.phys.nutrient.data.PeterPaul, lakename, year4, month, sampledate}\OperatorTok{:}\NormalTok{po4)}

\CommentTok{# find out the start and end dates of the dataset}
\KeywordTok{interval}\NormalTok{(NTL.phys.nutrient.data.PeterPaul}\OperatorTok{$}\NormalTok{sampledate[}\DecValTok{1}\NormalTok{], NTL.phys.nutrient.data.PeterPaul}\OperatorTok{$}\NormalTok{sampledate[}\DecValTok{23372}\NormalTok{])}
\end{Highlighting}
\end{Shaded}

\begin{verbatim}
## [1] 1984-05-27 UTC--2014-08-29 UTC
\end{verbatim}

\begin{Shaded}
\begin{Highlighting}[]
\KeywordTok{interval}\NormalTok{(}\KeywordTok{first}\NormalTok{(NTL.phys.nutrient.data.PeterPaul}\OperatorTok{$}\NormalTok{sampledate), }\KeywordTok{last}\NormalTok{(NTL.phys.nutrient.data.PeterPaul}\OperatorTok{$}\NormalTok{sampledate)) }\CommentTok{#more reprodusable}
\end{Highlighting}
\end{Shaded}

\begin{verbatim}
## [1] 1984-05-27 UTC--2014-08-29 UTC
\end{verbatim}

\subsection{SPLIT-APPLY-COMBINE}\label{split-apply-combine}

dplyr functionality, combined with the pipes operator, allows us to
split datasets according to groupings (function: \texttt{group\_by}),
then run operations on those groupings and return the output of those
operations. There is a lot of flexibility in this approach, but we will
illustrate just one example today.

\begin{Shaded}
\begin{Highlighting}[]
\NormalTok{NTL.PeterPaul.summaries <-}\StringTok{ }
\StringTok{  }\NormalTok{NTL.phys.nutrient.data.PeterPaul }\OperatorTok\StringTok{ }\CommentTok{#then}
\StringTok{  }\KeywordTok{filter}\NormalTok{(depth }\OperatorTok{==}\StringTok{ }\DecValTok{0}\NormalTok{) }\OperatorTok
\StringTok{  }\KeywordTok{group_by}\NormalTok{(lakename) }\OperatorTok
\StringTok{  }\KeywordTok{filter}\NormalTok{(}\OperatorTok{!}\KeywordTok{is.na}\NormalTok{(temperature_C) }\OperatorTok{&}\StringTok{ }\OperatorTok{!}\KeywordTok{is.na}\NormalTok{(tn_ug) }\OperatorTok{&}\StringTok{ }\OperatorTok{!}\KeywordTok{is.na}\NormalTok{(tp_ug)) }\OperatorTok\StringTok{ }\CommentTok{#wthout nas}
\StringTok{  }\KeywordTok{summarise}\NormalTok{(}\DataTypeTok{meantemp =} \KeywordTok{mean}\NormalTok{(temperature_C), }\CommentTok{#by group}
            \DataTypeTok{sdtemp =} \KeywordTok{sd}\NormalTok{(temperature_C), }
            \DataTypeTok{meanTN =} \KeywordTok{mean}\NormalTok{(tn_ug), }
            \DataTypeTok{sdTN =} \KeywordTok{sd}\NormalTok{(tn_ug), }
            \DataTypeTok{meanTP =} \KeywordTok{mean}\NormalTok{(tp_ug), }
            \DataTypeTok{sdTP =} \KeywordTok{sd}\NormalTok{(tp_ug))}

\KeywordTok{write.csv}\NormalTok{(NTL.PeterPaul.summaries, }\DataTypeTok{row.names =} \OtherTok{FALSE}\NormalTok{, }
          \DataTypeTok{file =}\StringTok{"./Data/Processed/NTL-LTER_Lake_Summaries_PeterPaul_Processed.csv"}\NormalTok{)}
\end{Highlighting}
\end{Shaded}

\subsection{ALTERNATIVE METHODS FOR DATA
WRANGLING}\label{alternative-methods-for-data-wrangling}

If you want to iteratively perform operations on your data, there exist
several options. We have demonstrated the pipe as one option. Additional
options include the \texttt{apply} function
(\url{https://www.rdocumentation.org/packages/base/versions/3.5.2/topics/apply})
and \texttt{for} loops
(\url{https://swcarpentry.github.io/r-novice-inflammation/15-supp-loops-in-depth/}).
These options are good options as well (again, multiple ways to get to
the same outcome). A word of caution: loops are slow. This may not make
a difference for small datasets, but small time additions will make a
difference with large datasets.


\end{document}
