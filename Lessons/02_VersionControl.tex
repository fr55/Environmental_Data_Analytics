\documentclass[]{article}
\usepackage{lmodern}
\usepackage{amssymb,amsmath}
\usepackage{ifxetex,ifluatex}
\usepackage{fixltx2e} % provides \textsubscript
\ifnum 0\ifxetex 1\fi\ifluatex 1\fi=0 % if pdftex
  \usepackage[T1]{fontenc}
  \usepackage[utf8]{inputenc}
\else % if luatex or xelatex
  \ifxetex
    \usepackage{mathspec}
  \else
    \usepackage{fontspec}
  \fi
  \defaultfontfeatures{Ligatures=TeX,Scale=MatchLowercase}
\fi
% use upquote if available, for straight quotes in verbatim environments
\IfFileExists{upquote.sty}{\usepackage{upquote}}{}
% use microtype if available
\IfFileExists{microtype.sty}{%
\usepackage{microtype}
\UseMicrotypeSet[protrusion]{basicmath} % disable protrusion for tt fonts
}{}
\usepackage[margin=2.54cm]{geometry}
\usepackage{hyperref}
\hypersetup{unicode=true,
            pdftitle={2: Workflow and Version Control},
            pdfauthor={Environmental Data Analytics \textbar{} Kateri Salk},
            pdfborder={0 0 0},
            breaklinks=true}
\urlstyle{same}  % don't use monospace font for urls
\usepackage{graphicx,grffile}
\makeatletter
\def\maxwidth{\ifdim\Gin@nat@width>\linewidth\linewidth\else\Gin@nat@width\fi}
\def\maxheight{\ifdim\Gin@nat@height>\textheight\textheight\else\Gin@nat@height\fi}
\makeatother
% Scale images if necessary, so that they will not overflow the page
% margins by default, and it is still possible to overwrite the defaults
% using explicit options in \includegraphics[width, height, ...]{}
\setkeys{Gin}{width=\maxwidth,height=\maxheight,keepaspectratio}
\IfFileExists{parskip.sty}{%
\usepackage{parskip}
}{% else
\setlength{\parindent}{0pt}
\setlength{\parskip}{6pt plus 2pt minus 1pt}
}
\setlength{\emergencystretch}{3em}  % prevent overfull lines
\providecommand{\tightlist}{%
  \setlength{\itemsep}{0pt}\setlength{\parskip}{0pt}}
\setcounter{secnumdepth}{0}
% Redefines (sub)paragraphs to behave more like sections
\ifx\paragraph\undefined\else
\let\oldparagraph\paragraph
\renewcommand{\paragraph}[1]{\oldparagraph{#1}\mbox{}}
\fi
\ifx\subparagraph\undefined\else
\let\oldsubparagraph\subparagraph
\renewcommand{\subparagraph}[1]{\oldsubparagraph{#1}\mbox{}}
\fi

%%% Use protect on footnotes to avoid problems with footnotes in titles
\let\rmarkdownfootnote\footnote%
\def\footnote{\protect\rmarkdownfootnote}

%%% Change title format to be more compact
\usepackage{titling}

% Create subtitle command for use in maketitle
\newcommand{\subtitle}[1]{
  \posttitle{
    \begin{center}\large#1\end{center}
    }
}

\setlength{\droptitle}{-2em}

  \title{2: Workflow and Version Control}
    \pretitle{\vspace{\droptitle}\centering\huge}
  \posttitle{\par}
    \author{Environmental Data Analytics \textbar{} Kateri Salk}
    \preauthor{\centering\large\emph}
  \postauthor{\par}
      \predate{\centering\large\emph}
  \postdate{\par}
    \date{Spring 2019}


\begin{document}
\maketitle

\subsection{LESSON OBJECTIVES}\label{lesson-objectives}

\begin{enumerate}
\def\labelenumi{\arabic{enumi}.}
\tightlist
\item
  Describe the components of version control using GitHub
\item
  Create a repository for this course using RStudio and GithHub
\item
  Apply version control skills to modify files and commit changes to
  local and remote repositories.
\end{enumerate}

\subsection{VERSION CONTROL}\label{version-control}

This semester, we are going to be incorporating the fundamentals of
\textbf{version control}, the process by which all changes to code,
text, and files are tracked. In this manner, we're also able to maintain
data and information to support collaborative projects, but to also make
sure your analyses are preserved.

Before coming to class, you were asked to create a GitHub.com account.
\textbf{GitHub} is the web hosting platform for maintaining our Git
repositories. Our version control system for the purposes of this course
is \textbf{Git}.

GitHub repositories come in two forms: public and private. All the
project repositories you will use for this course are public. There may
be situations in the future in which you may want to make use of private
repositories, in which data and progress are protected from public view.
Private repositories can be made public at a later date.

\subsection{LICENSING}\label{licensing}

Licensing is also important to specify who and how your code can be used
by others. Various license types are available for use with a decision
tree available \href{https://choosealicense.com/}{here}. For the
purposes of this work, we recommend a
\href{https://choosealicense.com/licenses/gpl-3.0/}{GNU General Public
License version 3}. This is a free and widely-trusted \emph{copyleft}
license. A copyleft is an arrangement whereby software or artist work
may be used, modified, and distributed freely on condition that anything
derived from it is bound by the same condition. Unless you plan to
charge a fee for use of your code, the GPLv3.0 comes highly recommended.

For more information about licensing your code, check out
\url{https://help.github.com/articles/licensing-a-repository/}

\subsection{FUNCTIONS IN GIT}\label{functions-in-git}

There are two ways to diverge code from a main project:
\textbf{branching} and \textbf{forking} A branch is a temporary segment
often used for development. We will not be making use of branching for
the purposes of this class, but it is worth spending some time reading
about them to see how they can be incorporated into collaborative
projects. A fork is a copy or clone of the main project. Forking allows
for greater oversight in the merging process and prevents changes being
committed to the main project.

\subsection{INTEGRATING GIT AND
RSTUDIO}\label{integrating-git-and-rstudio}

Make sure you have completed the necessary software installations of:

\textbf{R}

\textbf{RStudio}

\textbf{Git}, with GitHub user account

\textbf{LaTeX}

\textbf{Course Server}

\subsubsection{Forking and Cloning}\label{forking-and-cloning}

\begin{enumerate}
\def\labelenumi{\arabic{enumi}.}
\item
  Go to the following repository on GitHub:
  \url{https://github.com/KateriSalk/Environmental_Data_Analytics}
\item
  In the upper right corner, click the ``fork'' button. This will prompt
  you to create a copy of this repository in your own user account. This
  is called a fork of the original repository, which you can edit rather
  than simply download.
\item
  Clone your repository to your local drive using RStudio.
\end{enumerate}

\begin{itemize}
\item
  Copy the GitHub repository URL.
\item
  Click the ``Clone or download'' button and then the ``copy'' icon.
  Make sure the box header lists ``Clone with HTTPS'' rather than
  ``Clone with SSH.'' If not, click the ``Use HTTPS'' button and then
  copy the link.
\item
  Launch RStudio and select ``New Project'' from the File menu. Choose
  ``Version Control'' and ``Git.''"
\item
  Paste the repository URL and give your repository a name and a file
  path. The file path should be one that persists permanently, i.e.,
  either on the course server for lab computers or a dedicated space on
  your personal computer.
\end{itemize}

You now have tracked the course repository (i.e., the ``upstream''),
forked the repository into your own account (i.e., the ``remote''), and
cloned the repository to your drive (i.e., the ``local''). We will be
updating each of these repositories as the course progresses.

Note: Clone with SSH is an option as well. For a guide on creating SSH
keys, see the Duke Libraries online guide here and navigate to the
``Generate SSH keys'' section:
\url{https://git-rfun.library.duke.edu/outline.html\#generate_ssh_keys_in_advance_of_the_workshop}.

\subsubsection{Editing, Committing,
Pushing}\label{editing-committing-pushing}

\begin{enumerate}
\def\labelenumi{\arabic{enumi}.}
\item
  Navigate to the ``README.md'' file in the Files tab and open it.
\item
  Type your name after ``User:'' and save.
\item
  Now that you have edited a file, it should now appear in the Git tab.
  Click the box to the left of the file, where a check mark should now
  appear.
\item
  Press the ``Commit'' button. A new window should appear that shows the
  changes that have been made to the file.
\item
  Write a message detailing the edits you've made to the README file.
  You should always include a commit message to your commits so that
  your future self and/or your collaborators will know what changes were
  made. Click ``Commit''.
\item
  Click the green upward facing arrow: the ``Push'' button. Your remote
  repository is now up to date with your local repository.
\end{enumerate}

The commit command records the changes you've made to files in your
project. Commits are useful to track important milestones in your
progres. You should always make a commit at least once every time you
log onto a project (e.g., once daily). However, simply committing your
changes does not save those changes to your remote repository. This is
where push comes in. By pushing your new commits to the remote
repository, you are ensuring that your local and remote repositories
contain the same changes. You must have an internet connection to push,
but you can make several commits offline and push them when you regain
an internet connection.

Commits and pushes can also be done in the Terminal by typing commands
(i.e., the command line). Navigating to the Terminal tab will allow you
to type these commands. The relevant commands are:

\texttt{git\ commit}

\texttt{git\ push}

Note: files must be staged in order to use the command
\texttt{git\ commit}, and a commit message must still be used.

If you are using a Mac, you will need to take the following steps to
enter your commit message:

\begin{enumerate}
\def\labelenumi{\arabic{enumi}.}
\item
  press ``i'' to indicate insert
\item
  write your merge message
\item
  press ``esc'' to indicate you are exiting the insertion
\item
  write ``:wq'' to indicate write then quit
\item
  press enter
\end{enumerate}

\subsubsection{Pulling from the Remote}\label{pulling-from-the-remote}

If there are changes in your remote repository that you want to
incorporate into your local repository, you will need to pull them into
your local repository.

\begin{enumerate}
\def\labelenumi{\arabic{enumi}.}
\tightlist
\item
  Click the blue downward facing arrow: the ``Pull'' button. Your local
  repository is now up to date with your remote repository.
\end{enumerate}

\subsubsection{Pulling from the Upstream
Remote}\label{pulling-from-the-upstream-remote}

Your instructors will regularly update the course repository. However,
this will not update the repository that you've forked into your own
user account (i.e., remote) unless you tell it to do so.

Type the following commands into the Terminal window in RStudio (bottom
left)

\texttt{git\ remote\ add\ upstream\ https://github.com/KateriSalk/Environmental\_Data\_Analytics}

\texttt{git\ pull\ upstream}

An alternate to pull is fetch + merge:

\texttt{git\ fetch\ upstream}

\texttt{git\ merge\ upstream}

Note: these steps will update your local repository, but will not update
your remote. To bring your remote up to date, you will need to push your
changes.

\subsection{Merge}\label{merge}

Sometimes, files in your local and remote repositories will have changes
that are different. In this case, you will need to merge the files
together in a way that incorporates the changes in a manner that you
choose (merge). We will not be working directly with merge in this
course, but RStudio has a nice interface that will guide you through the
process of merging if it arises in the future.


\end{document}
