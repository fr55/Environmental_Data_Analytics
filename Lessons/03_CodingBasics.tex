\documentclass[]{article}
\usepackage{lmodern}
\usepackage{amssymb,amsmath}
\usepackage{ifxetex,ifluatex}
\usepackage{fixltx2e} % provides \textsubscript
\ifnum 0\ifxetex 1\fi\ifluatex 1\fi=0 % if pdftex
  \usepackage[T1]{fontenc}
  \usepackage[utf8]{inputenc}
\else % if luatex or xelatex
  \ifxetex
    \usepackage{mathspec}
  \else
    \usepackage{fontspec}
  \fi
  \defaultfontfeatures{Ligatures=TeX,Scale=MatchLowercase}
\fi
% use upquote if available, for straight quotes in verbatim environments
\IfFileExists{upquote.sty}{\usepackage{upquote}}{}
% use microtype if available
\IfFileExists{microtype.sty}{%
\usepackage{microtype}
\UseMicrotypeSet[protrusion]{basicmath} % disable protrusion for tt fonts
}{}
\usepackage[margin=2.54cm]{geometry}
\usepackage{hyperref}
\hypersetup{unicode=true,
            pdftitle={3: Coding Basics},
            pdfauthor={Environmental Data Analytics \textbar{} Kateri Salk},
            pdfborder={0 0 0},
            breaklinks=true}
\urlstyle{same}  % don't use monospace font for urls
\usepackage{color}
\usepackage{fancyvrb}
\newcommand{\VerbBar}{|}
\newcommand{\VERB}{\Verb[commandchars=\\\{\}]}
\DefineVerbatimEnvironment{Highlighting}{Verbatim}{commandchars=\\\{\}}
% Add ',fontsize=\small' for more characters per line
\usepackage{framed}
\definecolor{shadecolor}{RGB}{248,248,248}
\newenvironment{Shaded}{\begin{snugshade}}{\end{snugshade}}
\newcommand{\KeywordTok}[1]{\textcolor[rgb]{0.13,0.29,0.53}{\textbf{#1}}}
\newcommand{\DataTypeTok}[1]{\textcolor[rgb]{0.13,0.29,0.53}{#1}}
\newcommand{\DecValTok}[1]{\textcolor[rgb]{0.00,0.00,0.81}{#1}}
\newcommand{\BaseNTok}[1]{\textcolor[rgb]{0.00,0.00,0.81}{#1}}
\newcommand{\FloatTok}[1]{\textcolor[rgb]{0.00,0.00,0.81}{#1}}
\newcommand{\ConstantTok}[1]{\textcolor[rgb]{0.00,0.00,0.00}{#1}}
\newcommand{\CharTok}[1]{\textcolor[rgb]{0.31,0.60,0.02}{#1}}
\newcommand{\SpecialCharTok}[1]{\textcolor[rgb]{0.00,0.00,0.00}{#1}}
\newcommand{\StringTok}[1]{\textcolor[rgb]{0.31,0.60,0.02}{#1}}
\newcommand{\VerbatimStringTok}[1]{\textcolor[rgb]{0.31,0.60,0.02}{#1}}
\newcommand{\SpecialStringTok}[1]{\textcolor[rgb]{0.31,0.60,0.02}{#1}}
\newcommand{\ImportTok}[1]{#1}
\newcommand{\CommentTok}[1]{\textcolor[rgb]{0.56,0.35,0.01}{\textit{#1}}}
\newcommand{\DocumentationTok}[1]{\textcolor[rgb]{0.56,0.35,0.01}{\textbf{\textit{#1}}}}
\newcommand{\AnnotationTok}[1]{\textcolor[rgb]{0.56,0.35,0.01}{\textbf{\textit{#1}}}}
\newcommand{\CommentVarTok}[1]{\textcolor[rgb]{0.56,0.35,0.01}{\textbf{\textit{#1}}}}
\newcommand{\OtherTok}[1]{\textcolor[rgb]{0.56,0.35,0.01}{#1}}
\newcommand{\FunctionTok}[1]{\textcolor[rgb]{0.00,0.00,0.00}{#1}}
\newcommand{\VariableTok}[1]{\textcolor[rgb]{0.00,0.00,0.00}{#1}}
\newcommand{\ControlFlowTok}[1]{\textcolor[rgb]{0.13,0.29,0.53}{\textbf{#1}}}
\newcommand{\OperatorTok}[1]{\textcolor[rgb]{0.81,0.36,0.00}{\textbf{#1}}}
\newcommand{\BuiltInTok}[1]{#1}
\newcommand{\ExtensionTok}[1]{#1}
\newcommand{\PreprocessorTok}[1]{\textcolor[rgb]{0.56,0.35,0.01}{\textit{#1}}}
\newcommand{\AttributeTok}[1]{\textcolor[rgb]{0.77,0.63,0.00}{#1}}
\newcommand{\RegionMarkerTok}[1]{#1}
\newcommand{\InformationTok}[1]{\textcolor[rgb]{0.56,0.35,0.01}{\textbf{\textit{#1}}}}
\newcommand{\WarningTok}[1]{\textcolor[rgb]{0.56,0.35,0.01}{\textbf{\textit{#1}}}}
\newcommand{\AlertTok}[1]{\textcolor[rgb]{0.94,0.16,0.16}{#1}}
\newcommand{\ErrorTok}[1]{\textcolor[rgb]{0.64,0.00,0.00}{\textbf{#1}}}
\newcommand{\NormalTok}[1]{#1}
\usepackage{graphicx,grffile}
\makeatletter
\def\maxwidth{\ifdim\Gin@nat@width>\linewidth\linewidth\else\Gin@nat@width\fi}
\def\maxheight{\ifdim\Gin@nat@height>\textheight\textheight\else\Gin@nat@height\fi}
\makeatother
% Scale images if necessary, so that they will not overflow the page
% margins by default, and it is still possible to overwrite the defaults
% using explicit options in \includegraphics[width, height, ...]{}
\setkeys{Gin}{width=\maxwidth,height=\maxheight,keepaspectratio}
\IfFileExists{parskip.sty}{%
\usepackage{parskip}
}{% else
\setlength{\parindent}{0pt}
\setlength{\parskip}{6pt plus 2pt minus 1pt}
}
\setlength{\emergencystretch}{3em}  % prevent overfull lines
\providecommand{\tightlist}{%
  \setlength{\itemsep}{0pt}\setlength{\parskip}{0pt}}
\setcounter{secnumdepth}{0}
% Redefines (sub)paragraphs to behave more like sections
\ifx\paragraph\undefined\else
\let\oldparagraph\paragraph
\renewcommand{\paragraph}[1]{\oldparagraph{#1}\mbox{}}
\fi
\ifx\subparagraph\undefined\else
\let\oldsubparagraph\subparagraph
\renewcommand{\subparagraph}[1]{\oldsubparagraph{#1}\mbox{}}
\fi

%%% Use protect on footnotes to avoid problems with footnotes in titles
\let\rmarkdownfootnote\footnote%
\def\footnote{\protect\rmarkdownfootnote}

%%% Change title format to be more compact
\usepackage{titling}

% Create subtitle command for use in maketitle
\newcommand{\subtitle}[1]{
  \posttitle{
    \begin{center}\large#1\end{center}
    }
}

\setlength{\droptitle}{-2em}

  \title{3: Coding Basics}
    \pretitle{\vspace{\droptitle}\centering\huge}
  \posttitle{\par}
    \author{Environmental Data Analytics \textbar{} Kateri Salk}
    \preauthor{\centering\large\emph}
  \postauthor{\par}
      \predate{\centering\large\emph}
  \postdate{\par}
    \date{Spring 2019}


\begin{document}
\maketitle

\subsection{LESSON OBJECTIVES}\label{lesson-objectives}

\begin{enumerate}
\def\labelenumi{\arabic{enumi}.}
\tightlist
\item
  Develop familiarity with the form and function of the RStudio
  interface.
\item
  Apply basic functionality of R
\item
  Evaluate how basic practies in R contribute to best management
  practices for data analysis
\end{enumerate}

\subsection{TOUR OF RSTUDIO}\label{tour-of-rstudio}

Welcome to the RStudio interface. When you open RStudio, you will see
four panels:

\textbf{Source Code Editor} (top left) includes a tab structure to pull
up and edit R scripts and markdown documents.

\textbf{Console} (bottom left) interacts with R processes. R code is run
here. There is also a tab here called \texttt{Terminal} which will allow
you to access git functionality.

\textbf{Workspace Browser} (top right) holds the
\texttt{global\ environment} that is populated by analyses run in each R
session. There is also a \texttt{history} tab and a \texttt{git} tab.

\textbf{Notebook} (bottom right) holds tabs for \texttt{files},
\texttt{plots}, \texttt{packages}, and \texttt{help}. You will interact
with each of these functinalities, and we will explain each as they come
up.

More on the functinality of each of these panels as we move through this
lesson.

\subsection{R SCRIPTS AND R MARKDOWN
DOCUMENTS}\label{r-scripts-and-r-markdown-documents}

You are currently viewing an R Markdown document. This type of file
includes text chunks and R code chunks that can be viewed together. R
Markdown documents can also be ``knitted'' into a PDF or html format
(more on this later).

An R script file is similar to an R Markdown document, except it
\emph{only} includes R code. Any text that is included in an R script
that it not intended to be run as R code must be ``commented out'' so
that R does not interpret the text as code. We will practice with R
scripts later.

A few tips on navigating R Markdown and R script files. First: files can
be organized into sections. Sections can be expanded or collapsed as
desired via three options:

\begin{itemize}
\item
  On the lefthand side of the editor, you will see arrows to the right
  of the line numbers. Clicking on these arrows will collapse or expand
  the section. Try doing this with the ``R BASIC SYNTAX'' section below.
  When a section is collapsed, a double-arrow box will appear within the
  script. You can also click on this box directly to expand the section.
\item
  Navigating from the menu bar, the Edit menu will bring you to the
  option ``Folding''. This option can be especially helpful when you
  first open a file and decide if you want to navigate between sections
  or run sections sequentially.
\item
  Highlight a section of text, and then press \texttt{option} +
  \texttt{command} + \texttt{L} (Mac) or \texttt{alt} + \texttt{L}
  (Windows). Add the \texttt{shift} key to this combination to expand,
  or click the box.
\end{itemize}

\subsection{R BASIC SYNTAX}\label{r-basic-syntax}

(testing collapse/expand capabilities)

\subsubsection{R as a calculator}\label{r-as-a-calculator}

Below is a chunk of R code. You can run R code in several ways:

\begin{itemize}
\item
  Place your cursor on the line of R code that you want to run, then
  press \texttt{control\ +\ enter} (PC) or \texttt{command\ +\ enter}
  (Mac). Your R code should appear in the console, followed by any
  output generated by the code.
\item
  Highlight line(s) of R code, then press \texttt{control\ +\ enter}
  (PC) or \texttt{command\ +\ enter} (Mac). Your R code should appear in
  the console, followed by any output generated by the code. This is a
  good option if you want to run multiple lines of code at once.
\end{itemize}

\begin{Shaded}
\begin{Highlighting}[]
\CommentTok{# Basic math}
\DecValTok{1} \OperatorTok{+}\StringTok{ }\DecValTok{1}
\end{Highlighting}
\end{Shaded}

\begin{verbatim}
## [1] 2
\end{verbatim}

\begin{Shaded}
\begin{Highlighting}[]
\DecValTok{1} \OperatorTok{-}\StringTok{ }\DecValTok{1}
\end{Highlighting}
\end{Shaded}

\begin{verbatim}
## [1] 0
\end{verbatim}

\begin{Shaded}
\begin{Highlighting}[]
\DecValTok{2} \OperatorTok{*}\StringTok{ }\DecValTok{2}
\end{Highlighting}
\end{Shaded}

\begin{verbatim}
## [1] 4
\end{verbatim}

\begin{Shaded}
\begin{Highlighting}[]
\DecValTok{1} \OperatorTok{/}\StringTok{ }\DecValTok{2}
\end{Highlighting}
\end{Shaded}

\begin{verbatim}
## [1] 0.5
\end{verbatim}

\begin{Shaded}
\begin{Highlighting}[]
\DecValTok{1} \OperatorTok{/}\StringTok{ }\DecValTok{200} \OperatorTok{*}\StringTok{ }\DecValTok{30}
\end{Highlighting}
\end{Shaded}

\begin{verbatim}
## [1] 0.15
\end{verbatim}

\begin{Shaded}
\begin{Highlighting}[]
\DecValTok{5} \OperatorTok{+}\StringTok{ }\DecValTok{2} \OperatorTok{*}\StringTok{ }\DecValTok{3}
\end{Highlighting}
\end{Shaded}

\begin{verbatim}
## [1] 11
\end{verbatim}

\begin{Shaded}
\begin{Highlighting}[]
\NormalTok{(}\DecValTok{5} \OperatorTok{+}\StringTok{ }\DecValTok{2}\NormalTok{) }\OperatorTok{*}\StringTok{ }\DecValTok{3}
\end{Highlighting}
\end{Shaded}

\begin{verbatim}
## [1] 21
\end{verbatim}

\begin{Shaded}
\begin{Highlighting}[]
\CommentTok{# Common terms}
\KeywordTok{sqrt}\NormalTok{(}\DecValTok{25}\NormalTok{)}
\end{Highlighting}
\end{Shaded}

\begin{verbatim}
## [1] 5
\end{verbatim}

\begin{Shaded}
\begin{Highlighting}[]
\KeywordTok{sin}\NormalTok{(pi)}
\end{Highlighting}
\end{Shaded}

\begin{verbatim}
## [1] 1.224606e-16
\end{verbatim}

\begin{Shaded}
\begin{Highlighting}[]
\CommentTok{# Summary statistics}
\KeywordTok{mean}\NormalTok{(}\DecValTok{5}\NormalTok{, }\DecValTok{4}\NormalTok{, }\DecValTok{6}\NormalTok{, }\DecValTok{4}\NormalTok{, }\DecValTok{6}\NormalTok{)}
\end{Highlighting}
\end{Shaded}

\begin{verbatim}
## [1] 5
\end{verbatim}

\begin{Shaded}
\begin{Highlighting}[]
\KeywordTok{median}\NormalTok{(}\DecValTok{5}\NormalTok{, }\DecValTok{4}\NormalTok{, }\DecValTok{6}\NormalTok{, }\DecValTok{4}\NormalTok{, }\DecValTok{6}\NormalTok{)}
\end{Highlighting}
\end{Shaded}

\begin{verbatim}
## [1] 5
\end{verbatim}

\begin{Shaded}
\begin{Highlighting}[]
\CommentTok{# Conditional statements}
\DecValTok{4} \OperatorTok{>}\StringTok{ }\DecValTok{5}
\end{Highlighting}
\end{Shaded}

\begin{verbatim}
## [1] FALSE
\end{verbatim}

\begin{Shaded}
\begin{Highlighting}[]
\DecValTok{4} \OperatorTok{<}\StringTok{ }\DecValTok{5}
\end{Highlighting}
\end{Shaded}

\begin{verbatim}
## [1] TRUE
\end{verbatim}

\begin{Shaded}
\begin{Highlighting}[]
\DecValTok{4} \OperatorTok{!=}\StringTok{ }\DecValTok{5}
\end{Highlighting}
\end{Shaded}

\begin{verbatim}
## [1] TRUE
\end{verbatim}

\begin{Shaded}
\begin{Highlighting}[]
\DecValTok{4} \OperatorTok{==}\StringTok{ }\DecValTok{5}
\end{Highlighting}
\end{Shaded}

\begin{verbatim}
## [1] FALSE
\end{verbatim}

You may also choose to type or paste R code directly into the console.
This is not a recommended method, as it undermines the goals of
reproducibility (code is not saved). However, typing directly into the
console can be useful if you need to do something that is strictly
temporary (e.g., look at a summary of a dataset or determine the class
of a variable)

\subsubsection{Objects}\label{objects}

You can create R objects with an \emph{assignment} statement. The
indicator for an assigment is the \texttt{\textless{}-} symbol. A good
way to think about the meaning of an assignment statement is ``object
name (lefthand side) gets value (righthand side).''

A quick note: in many situations, a \texttt{=} sign will substitute for
a \texttt{\textless{}-}. Resist this temptation! This will be confusing
later, when \texttt{=} means something else.

\begin{Shaded}
\begin{Highlighting}[]
\NormalTok{x <-}\StringTok{ }\DecValTok{3}\OperatorTok{*}\DecValTok{4}
\end{Highlighting}
\end{Shaded}

Now, call up the object \texttt{x}. Notice that \texttt{x} has also just
shown up in your Environment tab.

\begin{Shaded}
\begin{Highlighting}[]
\NormalTok{x}
\end{Highlighting}
\end{Shaded}

\begin{verbatim}
## [1] 12
\end{verbatim}

\subsubsection{Naming}\label{naming}

R objects can be named with a combination of letters, numbers,
underscore (\texttt{\_}) and period (\texttt{.}). The best R object
names are \emph{informative}. Resist the temptation to call your R
object something convenient, like ``a'', ``b'', and so on. Calling your
R object something specific means that you can call up that object later
and have an idea of what it contains, with less need for specific
context.

Informative names are the first illustration of a common data management
recommendation: take the time to use best management practices at the
outset, and it will save you time in the long term.

Importantly, you may never call an R object ``data''. This word is
reserved for a specific function and may not be assigned as a name. To
work around this, many people call their R objects ``dat'', which is
another example of a less-than-ideal data management practice because it
is not informative.

Run the first line of code below. Then, type in ``long'' and press
\texttt{tab}. What happens?

What happens if there is a typo in your code? Type the following in the
R window: Long\_name\_for\_illustration longnameforillustration

\begin{Shaded}
\begin{Highlighting}[]
\NormalTok{long_name_for_illustration <-}\StringTok{ }\DecValTok{11}
\end{Highlighting}
\end{Shaded}

\subsubsection{Functions}\label{functions}

R functions are the major tool used in R. Functions can do virtually
unlimited things within the R universe, but each function requires
specific inputs that are provided under specific syntax. We will start
with a simple function that is built into R, \texttt{seq}

\begin{Shaded}
\begin{Highlighting}[]
\KeywordTok{seq}\NormalTok{(}\DecValTok{1}\NormalTok{, }\DecValTok{10}\NormalTok{)}
\end{Highlighting}
\end{Shaded}

\begin{verbatim}
##  [1]  1  2  3  4  5  6  7  8  9 10
\end{verbatim}

\begin{Shaded}
\begin{Highlighting}[]
\NormalTok{ten_sequence <-}\StringTok{ }\KeywordTok{seq}\NormalTok{(}\DecValTok{1}\NormalTok{, }\DecValTok{10}\NormalTok{)}
\NormalTok{ten_sequence}
\end{Highlighting}
\end{Shaded}

\begin{verbatim}
##  [1]  1  2  3  4  5  6  7  8  9 10
\end{verbatim}

\begin{Shaded}
\begin{Highlighting}[]
\NormalTok{(ten_sequence <-}\StringTok{ }\KeywordTok{seq}\NormalTok{(}\DecValTok{1}\NormalTok{, }\DecValTok{10}\NormalTok{))}
\end{Highlighting}
\end{Shaded}

\begin{verbatim}
##  [1]  1  2  3  4  5  6  7  8  9 10
\end{verbatim}

\begin{Shaded}
\begin{Highlighting}[]
\KeywordTok{seq}\NormalTok{(}\DecValTok{1}\NormalTok{, }\DecValTok{10}\NormalTok{, }\DecValTok{2}\NormalTok{) }\CommentTok{# from, to, by}
\end{Highlighting}
\end{Shaded}

\begin{verbatim}
## [1] 1 3 5 7 9
\end{verbatim}

\subsubsection{Getting help within R}\label{getting-help-within-r}

In many ways, the help functionality in R is limited by the fact that
you need to have a good understanding of specific functions for the help
to be useful. Google and Stack Overflow are often more helpful than the
help within R. We will practice those skills later.

For now, here are some ways to access the help tools in R:

\begin{itemize}
\item
  Within your R chunks in your editor, type in \texttt{??function}. This
  will bring up the help pane in the notebook, which you can then
  navigate through to find what you need.
\item
  In the console, type in \texttt{help(function)}. This will bring up
  the help pane in the notebook at the page for that function.
\item
  Navigate to the help pane in the notebook and type the function into
  the search bar.
\end{itemize}

\begin{Shaded}
\begin{Highlighting}[]
\NormalTok{??seq}
\end{Highlighting}
\end{Shaded}

\begin{verbatim}
## starting httpd help server ... done
\end{verbatim}

\subsubsection{Comments}\label{comments}

Within your R code, it is often useful to include notes about your
workflow. So that these aren't interpreted by the software as code,
precede the notes with a \texttt{\#} sign. Your editor will display this
comment as a different color to indicate it will not be run in the
console. Comments can be placed on their own lines or at the end of a
line of code.

\begin{Shaded}
\begin{Highlighting}[]
\CommentTok{# I am demonstrating a comment here. }

\DecValTok{1} \OperatorTok{+}\StringTok{ }\DecValTok{1} \CommentTok{# This is a simple math problem}
\end{Highlighting}
\end{Shaded}

\begin{verbatim}
## [1] 2
\end{verbatim}

\subsection{TIPS AND TRICKS}\label{tips-and-tricks}

\begin{itemize}
\item
  Spaces (generally) don't matter. One notable exception is that spaces
  within quotation marks \emph{do} matter.
\item
  Case matters
\item
  Parentheses and quotation marks appear in pairs when typed into
  RStudio.
\item
  When typing long names, use the \texttt{tab} key partway through the
  name to generate autocomplete options.
\item
  In the upper right corner of the editor is a button with multiple
  horizontal lines. Clicking this button will bring up the outline of
  the document. Headings in the outline are determined by how you've
  defined them in the document.
\end{itemize}


\end{document}
